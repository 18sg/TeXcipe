\documentclass[a4paper,10pt]{article}
\usepackage{recipe}

\begin{document}
	\section{Mains}
	\begin{recipe}{Mince \& Vegetable Base}
		A mince and vegetable based sauce which can be used (for example) as part of
		a Lasagne, Shepherd's Pie or Spaghetti Bolognese.
		
		\begin{ingredients}{5}
			\ingredient{Minced Beef\footnote{Use good stuff or it will all be
			            fat.}}{500g}
			\ingredient{Medium-Large Onion -- Chopped}{1}
			\ingredient{Medium Carrots -- Chopped (Optional)}{2}
			\ingredient{Pepper, Any Colour -- Chopped (Optional)}{1}
			\ingredient{Tinned Chopped Tomatoes\footnote{Use cheap stuff!}}{400g}
			\ingredient{Tinned Tomato Soup\footnote{Use cheap stuff!}}{400g}
			\ingredient{Mushrooms -- Sliced (Optional)}{175g}
			\ingredient{Dried Basil or Mixed Herbs}{1tsp}
			\ingredient{Ground Black Pepper}{1/8tsp}
		\end{ingredients}
		
		\begin{steps}
			\step Fry onion in tiny amount of olive oil (1 tsb) until just starting to
			      brown.
			
			\step Add mushrooms and fry for 2 minutes then take off heat.
			
			\step Chop peppers and carrots and add to the onion and mushrooms.
			
			\step Fry for 3-4 minutes.
			
			\step In a separate wok/frying pan cook mince.\footnote{If non-stick, no
			      oil is needed, otherwise use 1tsp of olive oil.}
			
			\step When mince is brown\footnote{If it is fatty, drain using a
			      colander}, add the vegetables to the mince.
			
			\step Add tinned soup and tomatoes and leave to bubble on low heat
			      stirring occasionally\footnote{About every 10 minutes} until carrots
			      are soft (approximately 10 minutes).\footnote{This mixture can be
			      kept in the state for a while so now is a good time to start making
			      the Cheese Sauce for a Lasagne.}
			\step Add herbs and black pepper.
		\end{steps}
	\end{recipe}
	
	\clearpage
	\begin{recipe}{Cheese Sauce}
		A simple cheese sauce which can be used to make Lasagne. Warning: stir and
		watch all the time as this can burn easily.
		
		\begin{ingredients}{5}
			\ingredient{Margarine}{50g}
			\ingredient{Plain Flour}{80g}
			\ingredient{Milk}{1 pint}
			\ingredient{Hard Cheese\footnote{e.g. Value Full-Flavour Cheddar, spare
			            hard ends of cheese.} -- Cubed}{150g}
		\end{ingredients}
		
		\begin{steps}
			\step Melt butter in saucepan on low heat.
			
			\step Add flour and stir until melted together.
			
			\step Add milk a bit at a time\footnote{About $\frac{1}{5}$ a pint at a
			      time} and stir until no lumps remain before adding more milk.
			
			\step Stir until mixture goes thick\footnote{It genuinely will!}.
			
			\step Add cheese and stir until melted and no lumps remain.
		\end{steps}
	\end{recipe}
	
	\clearpage
	\begin{recipe}{Lasagne}
		\begin{ingredients}{5}
			\ingredient{Dried Lasagne}{}
			\ingredient{Mince \& Vegetable Base}{}
			\ingredient{Cheese Sauce\footnote{If using a larger dish, use
			            $1\frac{1}{2}$ times as much.}}{}
			\ingredient{Grated Cheese}{}
		\end{ingredients}
		
		\begin{steps}
			\step Place mince base into a 30cm square Pyrex dish.
			
			\step Cover with 1 layer of Lasagne.
			
			\step Pour $\frac{1}{2}$ of cheese sauce on top.
			
			\step Add another layer of lasagne and cover with the rest of the cheese
			      sauce.
			
			\step Sprinkle with grated cheese.
			
			\step Place in a pre-heated 200\deg{c} oven for 35 minutes and serve.
			      \OR 200\deg{c} for 35 minutes and then up to $1\frac{1}{2}$ hours at
			      100\deg{c}\footnote{Cover with a loose lid of tinfoil if top gets
			      too brown}.
			      \OR Cool (and put in fridge) covered with foil/cling-film and use
			      the next day -- 200\deg{c} for 40 minutes.
			      \OR Cool and cover with foil/cling-film and freeze once cold.
			      Defrost in fridge for 24 hours or at room temperature for 12 hours.
			      Place in oven at 200\deg{c} for 40 minutes.
		\end{steps}
	\end{recipe}
\end{document}
